\documentclass[12pt]{article}
\usepackage{design_ASC}
\setlength\parindent{0pt} %% Do not touch this

\graphicspath{{./images/}}

%% -----------------------------
%% TITLE
%% -----------------------------
\title{Coursework X} %% Assignment Title

\author{Adam Peace\\
SN17008708\\
COMP00XY\\ %% Code and course name
\textsc{University College London}
}

\date{\today} %% Change "\today" by another date manually
%% -----------------------------
%% -----------------------------

%% %%%%%%%%%%%%%%%%%%%%%%%%%
\begin{document}
\setlength{\droptitle}{-5em}    
\maketitle

% %%%%%%%%%%%%%%%%%%%
\section*{Question 1}
{\bfseries You can write the questions here \ldots}

And your answers here \ldots

% %%%%%%%%%%%%%%%%%%%
\subsection*{Question 1.a}
{\bfseries Also, you can write your sub-questions into subsection levels.}

And again, your answers here \ldots

% %%%%%%%%%%%%%%%%%%%
\section*{Mathematical notation}

This template offers a set of customized mathematical symbols commonly used in FRED courses.

\begin{multicols}{2}
You can invoke the following symbols:
\begin{Verbatim}[frame=single, fontsize=\footnotesize]
\E(\cdot)             % Expectation
\V(\cdot)             % Variance
\Var(\cdot)           % Variance
\Cov(\cdot)           % Covariance
\Corr(\cdot)          % Correlation
\tr(\cdot)            % Trace
\rank(\cdot)          % Rank
\N(0,1)               % Normal
\op(\cdot)            % "little o" ope
\Op(\cdot)            % "big o" operator
\R^+                  % Real positive num
\La(y|\beta,\sigma^2) % Lagrange function
\end{Verbatim}
\footnotesize{
\begin{itemize}  \setlength\itemsep{0em}
    \item $\E(\cdot)$
    \item $\V(\cdot)$
    \item $\Var(\cdot)$
    \item $\Cov(\cdot)$
    \item $\Corr(\cdot)$
    \item $\tr(\cdot)$
    \item $\rank(\cdot)$
    \item $\N(0,1)$
    \item $\op(\cdot)$
    \item $\Op(\cdot)$
    \item $\R^{+}$
    \item $\La(y|\beta,\sigma^2)$
\end{itemize}}
\end{multicols}

Remember, these symbols have to be implemented into equation  environments, that means, with dollar symbols \$\ldots\$, or \texttt{equation}-environments.

\newpage 



% %%%%%%%%%%%%%%%%%%%
\section*{Code and scripts}

If you want to print the raw outcomes from any software, it is recommended the \texttt{Verbatim}-environment :
%% Verbatim -----------------
\begin{Verbatim}[frame = single, fontsize = \footnotesize]
ev = ev1 ~ ev2;
ev1 = sumc(ev[.,nest1]');
ev2 = sumc(ev[.,nest2]');
num = (ev1 .^ (k[1]-1)).*sumc(depm[.,nest1]') + (ev2 .^ (k[2]-1)).*sumc(depm[.,nest2]');
p = sumc((ev .* depm)').* num ./ ((ev1.^k[1])+(ev2.^k[2]));
\end{Verbatim}
%% Verbatim -----------------

\end{document}